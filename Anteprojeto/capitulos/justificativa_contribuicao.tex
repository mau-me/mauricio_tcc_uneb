\chapter{Justificativas e Contribuições}

Desenvolver um método de construção de árvores com base nas distâncias obtidas a partir da diferença do uso de códons contribuiria com a tarefa de classificação de cepas para a vigilância sanitária, especialmente na descoberta de novas cepas emergentes com potenciais pandêmicos. Ademais, é também importante dispor de alternativas à filogenia molecular atualmente utilizada, para gerar informações de outro ponto de vista e-ou para servir de referência aos métodos filogenéticos.
Os métodos atuais ainda demandam de um alto custo computacional, sendo assim, existe a necessidade de desenvolver outros mais baratos e que possam suportar o volume crescente de dados (sequências).
Sendo assim, o projeto visa apresentar um método que seja capaz de realizar classificações, com um custo computacional baixo, em relação a outros métodos, e que possa apresentar, do ponto de vista científico, alternativas de comparação com outras técnicas já existentes.

% Escrever a Motivação (por que) para a execução deste trabalho, destacando a relevância (social, econômica ou acadêmica) do mesmo. Descrever as razões pelas quais o projeto deve ser desenvolvido, quais as contribuições (acadêmicas e/ou científicas) para a área de conhecimento do projeto. Tal contribuição é assegurada pela utilidade do trabalho aos demais, pela contribuição cumulativa (ou seja, pelo que este acrescenta ao conjunto do conhecimento científico do tema), pelo ineditismo do tema ou da abordagem e pela contribuição à superação de lacunas no conhecimento.