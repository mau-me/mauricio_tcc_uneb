\chapter{Metodologia}

Um ponto importante para a obtenção dos objetivos deste trabalho está relacionada a definição da metodologia que servirá como alicerce. Com a proposta de desenvolver e validar um método de análise da evolução molecular de vírus com base no uso de códons, a metodologia escolhida para isso é o \ac{dsr}. Essa metodologia, proporciona um framework teórico e prático para a criação de artefatos inovadores, como métodos, modelos ou frameworks, visando resolver problemas específicos.\cite{peffers_dsr_2007}

Para a obtenção de sucesso ao utilizar o \ac{dsr} os seguintes passos serão seguidos:
\begin{enumerate}
  \item Identificação do problema e definição dos objetivos.
  \item Desenvolvimento do artefatos.
  \item Avaliação do artefato.
  \item Apresentar contribuições científicas.
\end{enumerate}

Também será utilizada analises quantitativas, ou seja, medidas estatísticas para mensurar e comparar os resultados obtidos.\\
A pesquisa quantitativa só tem sentido quando há um problema muito bem definido e há informação e teoria a respeito do objeto de conhecimento, entendido aqui como o foco da pesquisa e/ou aquilo que se quer estudar. Esclarecendo mais, só se faz pesquisa de natureza quantitativa quando se conhece as qualidades e se tem controle do que se vai pesquisar.\cite{da_silva_pesquisa_2014}

Os pontos a seguir serão realizados durante o desenvolvimento do projeto:
\begin{itemize}
  \item Coleta de sequências que serão utilizadas.
  \item Tratamento necessário dos dados.
  \item Avaliação de desempenho do modelo.
  \item Analise comparativa com modelos convencionais.
  \item Disponibilização do modelo como ferramenta web.
\end{itemize}

% Escrever detalhadamente a Metodologia (como será feito)  que será usada para execução do Trabalho. Em outras palavras, dizer como os objetivos geral e específicos serão atingidos, que ações serão executadas, que testes serão realizados, que indicadores serão usados, etc. Descrever também as tecnologias que serão utilizadas para o desenvolvimento do trabalho e como se pretende validar o projeto. Lembre de usar a fundamentação em estatística para trabalhos que tenham isto como requisito (pesquisas de campo, aplicação de questionários, estudos de caso, etc). Ao descrever as ações na metodologia não economize em detalhes. Quanto mais detalhadas forem as ações, mais fácil será estimar prazos exequíveis no cronograma.