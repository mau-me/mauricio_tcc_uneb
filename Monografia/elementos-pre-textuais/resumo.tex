Este trabalho teve como objetivo principal o desenvolvimento de um modelo para a análise de genomas virais, baseado no uso de códons. Essa ferramenta se propõe a ser um recurso significativo para a investigação da evolução de espécies, empregando sequências genômicas do SARS-COV-2 como base desse estudo. A implementação desse modelo visa proporcionar maior eficiência computacional aprimorada e alcançar resultados mais precisos. Adicionalmente, a ferramenta será capaz de apresentar visualizações gráficas dos resultados obtidos, simplificando a interpretação dos dados e auxiliando na tomada de decisões científicas. Espera-se que essa abordagem proporcione \textit{insights} valiosos sobre a evolução de espécies virais, contribuindo para o avanço da virologia e da genômica comparativa. Os resultados obtidos foram considerados satisfatórios, tendo em vista que o modelo teve sua implementação concluída de forma satisfatória, atendendo a todos os objetivos especificados.

% Separe as palavras-chave por ponto
\palavraschave{Bioinformática; Códons; Filogenia; Viral.}