%%%%%%%%%%%%%%%%%%%%%%%%%%%%%%%%%%%%%%%%%%%%%%%%%%%%%%%%%%%%%%%%%%%%%%%%%%%
%%%             DESCRICOES TEORICAS E FERRAMENTAS BASICAS               %%%
%%%%%%%%%%%%%%%%%%%%%%%%%%%%%%%%%%%%%%%%%%%%%%%%%%%%%%%%%%%%%%%%%%%%%%%%%%%

\setlength{\parskip}{0.3cm}

\chapter{Capítulo 2~-~Fundamentação Teórica}~\label{ch:fundamentacao}

Para iniciar, vamos testar novamente o uso de siglas: \gls{AA}. Observe que mesmo ao iniciar outro capítulo, a sigla não precisa ser definida novamente por extenso, pois já foi definida no capítulo anterior, além de estar definida na lista de siglas e abreviaturas.

A partir do capítulo~\ref{ch:fundamentacao} começa-se a apresentar a fundamentação teórica. Atenção ! Fundamentação teórica não é o mesmo que trabalhos relacionados ou estado da arte. Fundamentação teórica são conceitos teóricos - alguns mesmo antigos ou clássicos - que precisam ser compreendidos para entender a solução apresentada para o problema investigado.

Não há um modelo de divisão de capítulos. Alguns autores criam um único capítulo chamado `Fundamentação Teórica' e nele colocam todos os conteúdos divididos em seções, subseções, etc. Outros preferem criar capítulos diferentes para temas diferentes da fundamentação teórica. Enfim, não há uma regra rígida para isto. Deve ser definido em consenso com o orientador. O importante é ter bom senso para não criar vários capítulos minúsculos com 1 página ou menos. Se o que tem para falar de um tema é tão pouco, talvez valha à pena ter um único capítulo de fundamentação teórica com várias seções falando dos diversos temas que compõem sua fundamentação. Mas se cada tema tem muito a ser falado, pode ficar melhor dividi-los em capítulos diferentes.

Quanto a Trabalhos Relacionados ou Estado da Arte eles são as publicações recentes de pessoas que tentaram resolver o mesmo problema ou um problema similar. São o resultado da revisão sistemática de literatura feita no início do processo de TCC. Alguns autores optam por fazer a análise destes trabalhos dentro do capítulo de introdução. Outros optam por criar um capítulo denominado "Estado da Arte" ou "Revisão de Literatura" ou "Trabalhos Relacionados" e fazer esta análise lá. A decisão mais uma vez fica a critério de cada autor e deve ser tomada em consenso com o orientador. Se os Trabalhos Relacionados estiverem na introdução, certamente irá gerar uma introdução mais longa. Mas não há problema ou impedimento quanto a isto.

Uma vantagem de colocar na introdução é que logo no início da monografia já ficam claras as lacunas do estado da arte que seu trabalho buscará preencher. Quando feito num capítulo à parte isto só aparece depois, mas gera o benefício de não deixar a introdução tão extensa e por vezes cansativa para o leitor. Então é uma questão de decisão do autor e do orientador.

Quando colocado num capítulo à parte costuma vir antes dos capítulos de fundamentação teórica.