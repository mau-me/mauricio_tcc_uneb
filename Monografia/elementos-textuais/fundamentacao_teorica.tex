%%%%%%%%%%%%%%%%%%%%%%%%%%%%%%%%%%%%%%%%%%%%%%%%%%%%%%%%%%%%%%%%%%%%%%%%%%%
%%%             DESCRICOES TEORICAS E FERRAMENTAS BASICAS               %%%
%%%%%%%%%%%%%%%%%%%%%%%%%%%%%%%%%%%%%%%%%%%%%%%%%%%%%%%%%%%%%%%%%%%%%%%%%%%

\setlength{\parskip}{0.3cm}

\chapter{Fundamentação Teórica}~\label{ch:fundamentacao}
% FAZER
% Analisar e reescrever introdução da fundamentação teórica
A análise de genes virais está localizada em um campo interdisciplinar que exige uma sólida compreensão de diversos conceitos e técnicas. Dessa forma, abordaremos a fundamentação teórica necessária para a compreensão completa do projeto. Começaremos por abordar os princípios fundamentais da genética viral, discutindo o que é um genoma viral e o papel dos genes em vírus. Em seguida, examinaremos detalhadamente os códigos de codificação genética fornecida pela \gls{iupac}, que são essenciais para traduzir sequências de nucleotídeos em sequências de aminoácidos.

% FAZER
% Reescrever essa parte, tirando o DSR que está no outro capítulo
Este capítulo também destacará a importância da filogenética na classificação de genes virais e como as árvores filogenéticas são construídas com base em informações genéticas.

Para entender completamente os métodos e resultados apresentados nos capítulos subsequentes, é de suma importância absorver os conceitos evidenciados aqui. Pois o conhecimento teórico sólido proporcionará a base necessária para a observação crítica do desenvolvimento da nossa ferramenta.

\section{Biologia Molecular}
A Biologia Molecular é um ramo da biologia que lida e investiga os processos e mecanismos moleculares relacionados à estrutura, função e interações das biomoléculas presentes nos organismos vivos~\cite{alberts_biologia_2017}. Ou seja, consiste principalmente em estudar as relações entre os vários sistemas da célula, partindo da relação entre o \gls{dna}, \gls{rna} e a síntese de proteínas, e o modo como essas são reguladas.

Dessa forma, ao observar a estrutura do \gls{dna} apresentada na Figura~\ref{fig:estruturaDNA}, é possível notar que ela é uma molécula em forma de dupla hélice que carrega a informação genética em organismos vivos. Ela é composta por duas cadeias polinucleotídicas complementares enroladas em torno de um eixo central. Cada cadeia é formada por uma sequência de nucleotídeos, que consistem em uma pentose (a desoxirribose), um grupo fosfato e uma base nitrogenada que pode ser \gls{adenina}, \gls{timina}, \gls{citosina} ou \gls{guanina}. A maior parte dos organismos carrega suas informações genéticas no \gls{dna}, mas alguns vírus carregam essa informação no \gls{rna}, que possui as mesma bases do \gls{dna} só substituindo a \gls{timina} pela \gls{uracila}~\cite{genetica_peter_2017}.

\begin{figure}[htb]
  \centering
  \caption{Estrutura do DNA.}
  \includegraphics[scale=0.6]{figuras/estruturaDNA_02.pdf}
  \fonte{Retirada de \textit{\citeauthor{alberts_biologia_2017}}}~\label{fig:estruturaDNA}
\end{figure}

Além das bases nitrogenadas principais já apresentadas (\gls{adenina}, \gls{timina}, \gls{citosina} e \gls{guanina}), a \gls{iupac} que é uma organização não governamental internacional dedicada ao avanço da química, desenvolveu a codificação conhecida como \textit{IUPAC Nucleotide Code}, para representar de maneira padronizada as bases nitrogenadas encontradas nas moléculas de ácido nucleico~\cite{iupac_cornish_1985, iupac_hoppe_1986}. Também são apresentadas outras letras que simbolizam pares de bases ou misturas específicas, a letra ``N'' que é usada para retratar uma base desconhecida ou não especificada e os símbolos de ``.'' ou ``-'', conhecidos como ``GAP'', ou seja, representa onde há uma base identificável ou uma informação que é faltante~\cite{iupac_cornish_1985}, como é possível observar na Tabela~\ref{tab:iupacNucleotideCode} apresentada a seguir.

\begin{table}[htb]
  \caption{\textit{IUPAC Nucleotide Code}.}
  \begin{center}
    \begin{tabular}{c|c}
      \hline
      Base             & IUPAC Nucleotide Code    \\
      \hline
      Adenina          & A                        \\
      Citosina         & C                        \\
      Guanina          & G                        \\
      Timina           & T                        \\
      Uracila          & U                        \\
      A ou G           & R                        \\
      C ou T           & Y                        \\
      A ou C           & M                        \\
      G ou T           & K                        \\
      G ou C           & S                        \\
      A ou T ou G      & W                        \\
      C ou G ou T      & B                        \\
      A ou C ou T      & D                        \\
      A ou G ou T      & H                        \\
      C ou G ou A      & V                        \\
      A ou C ou G ou T & N                        \\
      GAP              & \textbf{.} ou \textbf{-} \\
      \hline
    \end{tabular}
  \end{center}
  \fonte{Adaptada de \citeauthor{iupac_cornish_1985}}\label{tab:iupacNucleotideCode}
\end{table}

% FAZER
% Corrigir/Complementar a definição de genoma e gene
% Comparar o genoma humano com o genoma viral em relação a quantidade de pares de bases
% https://www.genome.gov/genetics-glossary/Genome REF para imagens
Então, o conjunto completo de material genético contido em um organismo, seja ele um vírus, uma bactéria, um fungo, uma planta ou um animal é conhecido como genoma. O qual abrange todas as informações genéticas necessárias para o desenvolvimento, funcionamento e reprodução do organismo. Visto que ele é composto por sequências de \gls{dna} que carregam as instruções para a síntese de proteínas e regulam várias funções celulares~\cite{alberts_biologia_2017,genetics_benjamin_2016}. A sua análise desempenha um papel fundamental na genética, na biologia molecular e na compreensão da hereditariedade e da evolução~\cite{alberts_biologia_2017}.

Em virtude disso, quando as informações contidas no \gls{dna} são copiadas em uma molécula de \gls{rna}, acontece um processo conhecido como transcrição. E ele ocorre no núcleo das células e envolve a separação das duas fitas do \gls{dna} e o pareamento de nucleotídeos complementares para sintetizar uma molécula de \gls{mrna}. Portanto, o \gls{mrna} é uma cópia do DNA que carrega as bases nitrogenadas correspondente a um gene específico~\cite{alberts_biologia_2017}.
Após isso, ocorre o processo de tradução onde a sequência de bases nitrogenadas do \gls{mrna} é utilizada para sintetizar proteínas, e isso ocorre nos ribossomos presentes no citoplasma celular. Dessa forma, durante a tradução, o \gls{mrna} é lido em grupos de três bases nitrogenadas, chamados de códons~\cite{alberts_biologia_2017}. Os quais são seguimentos de três nucleotídeos consecutivos no RNA que correspondem a um aminoácido específico. Uma vez que o número de diferentes combinações de trincas formadas com quatro nucleotídeos é $4^3$, há 64 códons possíveis, entretanto, existem apenas 20 aminoácidos correspondentes, além dos de início (\textit{start codon}) e parada (\textit{stop codon}) como apresentado na Figura~\ref{fig:tabelaCodons}, conhecida como `código genético', e em muitos casos, códons correspondem ao mesmo aminoácido~\cite{alberts_biologia_2017}.
% Além disso, é durante o processo de tradução que a sequência de códons no RNA é utilizada para sintetizar proteínas. Durante a tradução, os códons são reconhecidos por moléculas de RNA transportador (tRNA) que trazem os aminoácidos correspondentes.
\begin{figure}[htb]
  \centering
  \caption{Tabela de Códons.}
  \includegraphics[scale=0.6]{figuras/tabelaCodons.pdf}
  \fonte{Adaptada de \citeauthor{openstax-genetic-code}}~\label{fig:tabelaCodons}
\end{figure}

Assim sendo, a relação entre os códons, o \gls{dna} e o \gls{rna} é crucial para a síntese de proteínas e a expressão genética. O sequenciamento do \gls{dna} e a identificação dos códons correspondentes permitem a inferência das sequências de aminoácidos nas proteínas codificadas por um determinado gene~\cite{alberts_biologia_2017}.
Cada códon especifica um aminoácido distinto. Os aminoácidos são transportados para o ribossomo por moléculas de \gls{trna}, que possuem um anticódon complementar ao códon do \gls{mrna}. À medida que o ribossomo percorre o \gls{mrna}, os aminoácidos são ligados em uma sequência específica, formando uma cadeia polipeptídica que será dobrada e modificada para se tornar uma proteína funcional~\cite{alberts_biologia_2017}.

% FAZER
% Sequenciamento genético (Como são obtidas as sequências) De forma bem sucinta
Portanto, para a completa elucidação dos processos mencionados previamente, é necessário aferir com precisão as sequências de nucleotídeos contidas em uma molécula de \gls{dna} ou \gls{rna}, empregando métodos confiáveis conhecidos como sequenciamento genético. As técnicas de sequenciamento predominantes abrangem o método de Sanger e o de \gls{ngs}, este último categorizado em dois grupos distintos: as plataformas de segunda geração, caracterizadas por leituras curtas (\textit{short reads}), e as de terceira geração, que se destacam por leituras longas (\textit{long reads})~\cite{nanopore_sequence_jain_2016,dna_sequence_sanger_1977,next_generation_sequence_goodwin_2016}.

\section{Vírus}

Os vírus são agentes infecciosos que possuem uma estrutura viral que varia entre os seus diferentes tipos, mas que de modo geral é composta por uma cápsula proteica chamada capsídeo, a qual envolve o material genético, que pode ser \gls{dna} ou \gls{rna}~\cite{david_virology_2022}, podendo apresentar diferentes formas, como hélices, icosaedros ou outras mais complexas. Além do capsídeo, alguns vírus possuem uma camada lipídica chamada envelope viral, que é derivada da membrana da célula hospedeira e contém glicoproteínas que são importantes para a entrada do vírus nas células hospedeiras~\cite{david_virology_2022, alberts_molecular_2002}.
O ciclo e vida viral é conjunto de etapas que um contemplam o processo de reproduzir e infectar novas células. Assim como na estrutura, nos ciclos também podem ocorrer variações, mas geralmente envolvem as seguintes etapas~\cite{alberts_molecular_2002}:

\begin{enumerate}
  \item \textbf{Adsorção:} o vírus se liga especificamente a receptores na superfície da célula hospedeira.
  \item \textbf{Penetração:} o vírus é internalizado na célula hospedeira, liberando seu material genético.
  \item \textbf{Replicação e síntese de proteínas virais:} o material genético viral é transportado para os ribossomos da célula hospedeira, replicado e transcrito em moléculas de \gls{mrna}, que são utilizados para a síntese de proteínas virais.
  \item \textbf{Montagem:} as proteínas virais se unem para formar novas partículas virais.
  \item \textbf{Liberação:} as novas partículas virais são liberadas da célula hospedeira, para a montagem de novos vírus e para a modificação do ambiente celular para garantir a sua replicação.
        % \item \textbf{Liberação:} as novas partículas virais são liberadas da célula hospedeira, podendo ocorrer por lise celular ou por brotamento
\end{enumerate}

\subsection{SARS-CoV-2}

O \gls{sarscov2} é um vírus da família \textit{Coronaviridae}, que causa a doença chamada \gls{covid19}. Ele foi identificado pela primeira vez em dezembro de 2019 na cidade de Wuhan, na província de Hubei, na China, e desde então se espalhou para todo o mundo, resultando em uma pandemia~\cite{zhu_novel_2020,wu_coronavirus_2020}.

Visto que o \gls{sarscov2} possui uma estrutura viral (apresentada na Figura~\ref{fig:estruturaCoronavirus}) característica dos coronavírus, há em sua composição uma partícula viral esférica, com um envelope lipídico que envolve seu material genético. E, além de incluir proteínas de espículas na sua superfície, conhecidas como proteína spike (S). Ele apresenta também proteínas de membrana (M), envelope (E), nucleocapsídeo (N) e o seu \gls{rna} genômico que é composto por uma única cadeia de sentido positivo conforme ilustrada na Figura~\ref{fig:genomaCoronavirus}, que contém todas as informações genéticas necessárias para a replicação e síntese de proteínas virais~\cite{covid19_cascella_2020}.

\begin{figure}[htb]
  \centering
  \caption{Estrutura do coronavírus.}
  \includegraphics[scale=0.8]{figuras/estruturaSarsCov2.pdf}
  \fonte{Retirada de \textit{\citeauthor{li_coronavirus_2020}}}~\label{fig:estruturaCoronavirus}
\end{figure}

\begin{figure}[htb]
  \centering
  \caption{Genoma de RNA de cadeia simples do SARS-CoV-2.}
  \includegraphics[scale=0.6]{genoma_sarscov2.png}
  \fonte{Adaptada de \citeauthor{covid19_cascella_2020}}~\label{fig:genomaCoronavirus}
\end{figure}

% SPIKE
Sendo assim, é notório que a proteína Spike (S) do \gls{sarscov2} desempenha um papel de suma importância na infecção de células hospedeiras, pois trata-se de uma glicoproteína que forma estruturas semelhantes a espículas na superfície do vírus, conferindo-lhe uma aparência coroada. E, apesar de ser o alvo principal das respostas imunes do hospedeiro, ela é fundamental para a ligação do vírus às células humanas e sua subsequente entrada. Além disso, ela é composta por três domínios principais, a saber: o domínio de ligação ao receptor (RBD -~\textit{Receptor-Binding Domain}), o domínio de fusão (FD -~\textit{Fusion Domain}) e o domínio N-terminal (NTD -~N-\textit{Terminal Domain}). Sendo que o RBD é particularmente importante, pois é responsável pela interação com o receptor da enzima conversora de angiotensina 2 (ACE2) nas células hospedeiras humanas. Essa interação é essencial para a entrada do vírus nas células~\cite{covid19_cascella_2020}.
% Apresentar imagem específica da SPIKE

Em virtude disso, a estrutura da proteína Spike é altamente dinâmica e pode mudar de conformação para facilitar a fusão da membrana viral com a membrana da célula hospedeira, permitindo assim a entrada do vírus. Essa capacidade de mudança conformacional a torna um alvo promissor para o desenvolvimento de vacinas e terapias antivirais. Portanto, estudos detalhados da proteína Spike são essenciais para compreender a patogenicidade do vírus SARS-CoV-2 e para o desenvolvimento de estratégias terapêuticas eficazes. Além disso, mutações na proteína Spike têm sido identificadas como uma das principais causas de variantes, o que destaca ainda mais a importância de sua investigação contínua~\cite{covid19_cascella_2020}.
% SPIKE

\subsubsection{Nomenclaturas de Linhagens do SARS-CoV-2}
A nomenclatura das linhagens do \gls{sarscov2} é uma parte crucial na classificação e rastreamento das diferentes variantes do vírus, sendo que as duas mais usadas para descrever essas linhagens são a \gls{pango} e a da \gls{who}. Elas tem como objetivo, descrever as diferentes variantes do vírus com base em suas características genéticas e filogenéticas~\cite{pango_rambaut_2020,who_variants}.

Sendo assim, a nomenclatura \gls{pango} é uma abordagem baseada na filogenia para nomear e rastrear as linhagens do \gls{sarscov2}. Ela busca atribuir um nome único a cada tipo com base em sua posição na árvore filogenética do vírus, permitindo uma identificação clara das diferentes espécies ajudando na compreensão de como está ao longo do tempo~\cite{pango_rambaut_2020}.

Enquanto que a \gls{who} também desenvolveu sua própria nomenclatura para classificar as variantes do \gls{sarscov2}, utilizando letras gregas em ordem alfabética (Alpha, Beta, Gamma, Delta, etc.) e foi aplicada para evitar a estigmatização de locais geográficos ou populações~\cite{who_variants,covid_current_chenchula_2022,covid_raman_2021}. Além disso, a \gls{who} também dividiu as variantes em 3 (três) grupos distintos: \gls{vum}, \gls{voi} e \gls{voc}~\cite{covid19_cascella_2020,covid_raman_2021}.

A \gls{vum} é um termo usado para sinalizar às autoridades de saúde pública que uma variante do SARS-CoV-2 pode exigir atenção e monitoramento priorizados. O principal objetivo desta categoria é investigar se esta variante (e outras intimamente relacionadas com ela) pode representar uma ameaça adicional à saúde pública global em comparação com outras variantes em circulação.
Já a \gls{voi}, é um termo usado para descrever uma variante do SARS-CoV-2 com alterações que afetam o comportamento do vírus ou seu impacto potencial na saúde humana. Isto pode incluir, por exemplo, a sua capacidade de propagação, a sua capacidade de causar doenças graves ou a facilidade com que pode ser detectada ou tratada. Um VOI também pode ser identificado porque tem uma maior capacidade de propagação quando comparado com outras variantes em circulação, sugerindo um potencial risco emergente para a saúde pública global. Por fim, a \gls{voc} é um termo que descreve uma variante do SARS-CoV-2 que atende à definição de \gls{voi}, mas também atende a pelo menos um dos seguintes critérios quando comparado com outras variantes~\cite{who_variants}:
\begin{itemize}
  \item pode causar uma mudança prejudicial na gravidade da doença.
  \item   pode ter um impacto substancial na capacidade dos sistemas de saúde de prestar cuidados a pacientes com COVID-19 ou outras doenças e, portanto, exigir grandes intervenções de saúde pública.
  \item há uma diminuição significativa na eficácia das vacinas disponíveis na proteção contra doenças graves.
\end{itemize}

A tabela~\ref{tab:nomenclaturaPangoWho} apresentada a seguir, contém as \gls{voc} \textit{Alpha, Beta, Gamma, Delta e Omicron}\footnote{Essa variante possui subvariantes como BA.1, BA.2, BA.3, BA.4 e BA.5}, com as suas respectivas nomenclaturas \gls{pango} e \gls{who}~\cite{covid19_cascella_2020}:

\begin{table}[htb]
  \caption{Variantes VOC e suas Nomenclaturas \textit{PANGO e WHO}.}
  \begin{center}
    \begin{tabular}{c|c}
      \hline
      Nomenclatura WHO & Nomenclatura PANGO \\
      \hline
      Alpha            & B.1.1.7            \\
      Beta             & B.1.351            \\
      Gamma            & P.1                \\
      Delta            & B.1.617.2          \\
      Omicron          & B.1.1.529          \\
      \hline
    \end{tabular}
  \end{center}
  \fonte{O Autor}\label{tab:nomenclaturaPangoWho}
\end{table}

\section{Bioinformática}
% Alinhamento de Sequências, Análise de Genomas Virais, Análise de Genomas Virais, Análise de Expressão Gênica, Ferramentas de Bioinformática (Algoritmos de Alinhamento e Algoritmos de reconstrução de árvores)
A bioinformática é um campo interdisciplinar que aplica técnicas de ciência da computação e estatística para entender e interpretar dados biológicos. Essa área surgiu com o advento das tecnologias de sequenciamento de DNA e tem se expandido para abranger diversos aspectos da biologia molecular e genômica~\cite{bioinformatica_verli_2014}.

Tendo em vista a grande quantidade de informações geradas, surgiram diversos bancos de dados genéticos. Os bancos de dados genéticos desempenham um papel crucial na bioinformática, fornecendo uma vasta coleção de informações sobre sequências genéticas, expressão gênica, variantes genéticas, estruturas de proteínas e muito mais. Entre os principais estão o GenBank, que é gerenciado pela \gls{ncbi} e o \gls{bvbrc}, que é um centro de recursos de bioinformática dedicado ao estudo e análise de bactérias e vírus. O site também disponibiliza uma coleção abrangente de banco de dados, incluindo sequências genômicas, anotações funcionais, informações de expressão gênica e estruturas tridimensionais~\cite{ncbi, bvbrc}.

Os bancos de dados genéticos disponibilizam as sequências em diversos formatos diferentes. Entre esses, um dos principais é o formato {FASTA}. O formato {FASTA} é uma notação amplamente utilizado na bioinformática para representar sequências biológicas, como sequências de DNA, RNA ou proteínas. Este formato simples e legível por humanos facilita o armazenamento, a transferência e a análise de dados de sequências. Um arquivo {FASTA} típico consiste em duas partes principais: a linha de descrição (cabeçalho) e a sequência propriamente dita. A linha de descrição começa com o caractere `>' seguido pelo nome ou identificador da sequência, e opcionalmente, uma breve descrição. A sequência é listada imediatamente abaixo, podendo ser quebrada em linhas para facilitar a leitura~\cite{fasta_improved_pearson_1988}.

% FAZER Verificar se está correto ou não e adicionar as referências
O alinhamento de sequências é uma tarefa fundamental na bioinformática, permitindo comparar e identificar similaridades entre sequências biológicas. Existem dois tipos principais de alinhamento de sequências. O alinhamento de sequência global, que busca alinhar sequências completas, muitas vezes utilizado para comparar genomas inteiro e o alinhamento de sequência local, que concentra-se em identificar regiões específicas de semelhança, frequentemente empregado para encontrar homologias em genes ou proteínas~\cite{similarities_needleman-wunsch_1970,bioinformatics_david_2004}.

\section{Filogenia}
% FAZER
% Escrever sobre Modelos Evolutivos
% GTR Generalized time Reversible

A filogenia é uma disciplina da biologia que estuda as relações evolutivas entre organismos, buscando reconstruir a história evolutiva e a ancestralidade comum. A filogenética molecular é uma abordagem utilizada para inferir a filogenia com base em informações moleculares, como sequências de \gls{dna}, \gls{rna} e proteínas~\cite{felsenstein_inferring_2004}.

A construção de árvores filogenéticas é um aspecto fundamental da filogenética molecular. Existem vários métodos utilizados para construir árvores filogenéticas, que podem ser classificados em dois grupos principais: métodos baseados em distância e métodos baseados em caracteres.
Os métodos baseados em distância medem a similaridade ou a dissimilaridade entre sequências moleculares e constroem árvores filogenéticas com base nessas medidas~\cite{saitou_neighbor_1987,felsenstein_inferring_2004}. Alguns exemplos de métodos baseados em distância incluem o método de junção de vizinhos e o método de Mínima Evolução (ME).
Por outro lado, os métodos baseados em caracteres analisam as mudanças nos caracteres moleculares ao longo do tempo para inferir as relações filogenéticas. Exemplos de métodos baseados em caracteres são o método de Máxima Parcimônia (MP) e o método de Inferência Bayesiana~\cite{swofford_phylogenetic_1996}.

% Abordagens anteriores em ordem cronológica
Ao longo dos anos, vários métodos utilizados para análise filogenética foram desenvolvidos, logo após, serão apresentados alguns dos principais e amplamente utilizados:

\begin{enumerate}
  \item \textbf{Método de reconstrução de árvore filogenética de distância (1957)}: Esse método é baseado na construção de árvores filogenéticas a partir de uma matriz de distâncias que quantifica a diferença evolutiva entre diferentes sequências. A árvore é construída de modo que as sequências mais semelhantes estejam mais próximas umas das outras. Esse método é amplamente utilizado em análises filogenéticas e é uma das técnicas mais antigas~\cite{sokal_statistical_method_1958}.
  \item \textbf{Método de máxima parsimônia (1966)}: A máxima parsimônia busca a árvore filogenética mais simples, ou seja, aquela que requer o menor número de mudanças evolutivas para explicar as sequências observadas. Esse método é baseado no princípio de que a evolução segue o caminho mais econômico, evitando mudanças desnecessárias~\cite{fitch_toward_definition_1971}.
  \item \textbf{Método de máxima verossimilhança (1981)}: O método de máxima verossimilhança estima a árvore filogenética que maximiza a probabilidade de observar as sequências dadas, dadas as hipóteses filogenéticas. Ele é baseado na modelagem estatística da evolução molecular e é considerado um dos métodos mais precisos para a reconstrução de árvores filogenéticas~\cite{felsenstein_evolutionary_tree_1981}.
  \item \textbf{Método de junção de vizinhos (1987)}: É uma técnica de construção de árvore filogenética que se baseia em uma abordagem de aglomeração hierárquica. Ele é amplamente utilizado para criar árvores filogenéticas a partir de matrizes de distância, representando a proximidade evolutiva entre sequências ou espécies. Esse método é especialmente útil para análises de grandes conjuntos de dados e é conhecido por sua eficiência computacional~\cite{saitou_neighbor_1987}. % MÉTODO UTILIZADO NO AGUA!? Junção de Vizinhos / Neighbor-Joining
  \item \textbf{Método de inferência bayesiana (2001)}: A inferência bayesiana combina informações a priori com dados observados para estimar a árvore filogenética mais provável. Ela se baseia no Teorema de Bayes e permite incorporar informações prévias sobre as relações filogenéticas. Esse método é particularmente útil quando se dispõe de conhecimento prévio sobre as relações entre as espécies~\cite{huelsenbeck_bayesian_inference_2001}.
  \item \textbf{Método de coalescência (2004)}: O método de coalescência, também conhecido como filogenia de coalescência, aborda a filogenia a partir do ponto de vista do ancestral comum mais recente. Ele modela a história da população ancestral e como as sequências evoluíram a partir dessa população. Esse método é especialmente útil para analisar sequências de genes individuais~\cite{kingman_coalescent_1982}.
  \item \textbf{Método de redes filogenéticas (2005)}: As redes filogenéticas são uma extensão das árvores filogenéticas que permitem representar relacionamentos mais complexos, como reticulações ou eventos de hibridização. Elas são úteis quando as relações entre as espécies não podem ser adequadamente representadas por uma árvore simples~\cite{huson_phylogenetic_networks_2006}.
  \item \textbf{Método de filogenia de genoma inteiro (2010)}: Esse método se concentra na análise comparativa de genomas completos para inferir relações filogenéticas. Ele utiliza informações genômicas de alta resolução, como sequências de genes e elementos regulatórios, para construir árvores filogenéticas que refletem a evolução das espécies~\cite{eisen_horizontal_gene_transfre_2000}.
\end{enumerate}

\section{Machine Learning}
\textit{Machine learning} é uma subárea da inteligência artificial que se concentra no desenvolvimento de algoritmos e modelos que permitem a um sistema aprender a partir de dados e realizar tarefas específicas sem ser explicitamente programado~\cite{deeplearning_goodfellow_2016}. Dentro do campo do \textit{machine learning}, existem duas categorias principais de aprendizado: supervisionado e não supervisionado. Neste contexto, abordaremos a segunda categoria, com ênfase nos modelos não supervisionados.

O aprendizado não supervisionado é uma abordagem de \textit{machine learning} na qual o algoritmo é treinado em dados não rotulados, ou seja, dados que não têm rótulos ou categorias previamente atribuídos. O objetivo dele é explorar a estrutura e os padrões subjacentes aos dados sem orientação externa. Por certo, isso torna esse campo adequado para tarefas em que a natureza dos dados é desconhecida, e os padrões emergentes devem ser identificados~\cite{machine_learning_bishop_2006}.

Um dos principais tipos de tarefa no aprendizado não supervisionado é o agrupamento (\textit{clustering}). Nessa atividade, o algoritmo identifica grupos ou \textit{clusters} de dados que compartilham características semelhantes. O objetivo é reunir dados de acordo com suas propriedades intrínsecas, sem conhecimento prévio das categorias. Algoritmos de \textit{clustering}, como o K-Means e o Hierarchical Clustering, são amplamente utilizados em campos como biologia, processamento de imagem, análise de dados e muito mais~\cite{machine_learning_bishop_2006}.

Dessa forma, é visível que o aprendizado não supervisionado é fundamental em diversas aplicações. Pois, como na biologia, por exemplo, algoritmos de \textit{clustering} podem ser usados para identificar grupos de genes que são coexpressos, revelando padrões de regulação genética. Já em finanças, a redução de dimensionalidade pode ser aplicada para entender a relação entre diferentes ativos financeiros. Na área de processamento de linguagem natural, ele é usado para detectar tópicos em grandes volumes de texto~\cite{bioinformatics_david_2004}.


Apesar de sua versatilidade, o aprendizado não supervisionado também apresenta desafios. Pois a interpretação dos resultados pode ser complexa, uma vez não há rótulos de classe para validar as descobertas. Além disso, a escolha de hiperparâmetros e a avaliação da qualidade do agrupamento ou da redução de dimensionalidade podem ser complicadas.
Em resumo, o aprendizado não supervisionado desempenha um papel elementar no campo do \textit{machine learning}, permitindo a extração de informações valiosas de dados não rotulados. Ademais, sua capacidade de encontrar estrutura oculta nos dados é crucial em uma variedade de domínios, tornando-o uma ferramenta poderosa na análise e interpretação de informações intricadas~\cite{learning_kernels_scholkopf_2002}.

\section{Trabalhos Correlatos}

Na busca de trabalhos relacionados, vários métodos foram encontrados, e a seguir são apresentados.

O método de \gls{ml} (ou \textit{Maximum Likelihood}), não é exclusivo da filogenia, mas sim uma abordagem estatística. Visto que a sua aplicação em filogenia consiste em avaliar a probabilidade de que o modelo de evolução escolhido gere os dados observados, que são por exemplo, características de um organismo. Essa proposta foi utilizada nos seguintes trabalhos:
\begin{itemize}
  \item \textit{\citeauthor{behl_threat_2022}}
  \item \textit{\citeauthor{fall_genetic_diversity_2021}}
  \item \textit{\citeauthor{shabbir_comprehensive_2020}}
  \item \textit{\citeauthor{hudu_hepatitis_2018}}
  \item \textit{\citeauthor{sallard_tracing_2021}}
  \item \textit{\citeauthor{paez-espino_diversity_evolution_2019}}
  \item \textit{\citeauthor{tang_evolutionary_2021}}
  \item \textit{\citeauthor{cho_analysis_2022}}
\end{itemize}

Já em~\textit{\citeauthoronline{yin_systematic_2019}} e \textit{\citeauthoronline{bedoya-pilozo_molecular_epidemiology_2018}}, foi usada a inferência bayesiana, que é fundamentada no teorema de Bayes, que permite a atualização das probabilidades a priori para probabilidades a posteriori à medida que novas evidências são incorporadas.

Além desses,~\textit{\citeauthor{potdar_phylogenetic_2021}} utilizou a junção de vizinhos, que é baseado em uma abordagem heurística que visa construir uma árvore filogenética a partir de uma matriz de distância entre as sequências estudadas. O trabalho de~\citeauthoronline{lichtblau_alignment-free_2019} expos o \textit{Frequency Chaos Game Representation} e~\textit{\citeauthoronline{kim_ngs_2022}} a floresta aleatória. Por fim,~\textit{\citeauthoronline{dimitrov_updated_2019}} comparou três modelos para reconstrução de árvores filogenéticas: junção de vizinhos; \gls{ml} e inferência bayesiana.