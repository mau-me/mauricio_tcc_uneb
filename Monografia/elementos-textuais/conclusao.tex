\chapter{Considerações finais}

Neste estágio avançado do projeto, podemos destacar que alcançamos marcos significativos no desenvolvimento de nossa ferramenta de análise de genes virais com base no uso de codons para classificação não supervisionada. Ao longo deste processo, enfrentamos desafios técnicos e científicos, e estamos progredindo de maneira consistente em direção à conclusão do projeto.

Primeiramente, concluímos com sucesso a montagem do dataset, um dos passos fundamentais para o desenvolvimento do nosso método. Utilizando técnicas de download de forma automatizada, para baixar as sequências classificadas de genoma completo do vírus a partir da base de dados pública \gls{bvbrc}, fomos capazes de construir um conjunto de dados abrangente e representativo. Além disso, implementamos um procedimento de filtragem rigoroso para selecionar apenas sequências únicas, garantindo a qualidade dos dados utilizados em nossa análise.

Além disso, desenvolvemos um conjunto de etapas para o processamento das sequências genéticas, incluindo o alinhamento com a ferramenta Minimap2, a extração de genes de interesse usando o conceito de `isca' e a remoção de sequências duplicadas. Essas etapas são cruciais para preparar os dados de maneira adequada para análise subsequente.

O desenvolvimento do modelo de classificação não supervisionada também foi concluído com sucesso, permitindo-nos traduzir sequências de DNA em codons e extrair códigos únicos com base nas posições com codons distintos. Isso estabelece as bases para o nosso método de agrupamento e classificação.

Neste ponto, estamos nos estágios finais do projeto, com apenas uma etapa pendente. A finalização de um script que será utilizado para a comparação entre o método desenvolvido e um método tradicional está em andamento. Esta etapa é crucial para avaliar a eficácia e a precisão de nossa abordagem em relação às técnicas clássicas de filogenética. Após a finalização deste script, realizaremos a execução completa do método com o dataset completo, possibilitando uma avaliação abrangente.

Além disso, continuaremos a monitorar de perto o custo computacional, prestando atenção especial ao tempo necessário para classificar as sequências. Essa informação será valiosa para avaliar a escalabilidade e a eficiência do nosso método.

Em resumo, estamos avançando de maneira sólida e estruturada em direção aos objetivos de nosso projeto. Com a conclusão iminente da etapa final e a execução com o dataset completo, estamos ansiosos para avaliar os resultados e contribuir para a pesquisa em genética viral e análise filogenética.

Esta conclusão reflete o progresso atual do projeto, destacando as realizações e as etapas futuras para atingir os objetivos finais.

% Neste capítulo são apresentadas duas coisas: Conclusões e Trabalhos Futuros. Aqui devem ser apresentadas a que conclusões se pode chegar a partir dos resultados apresentados e analisados no capítulo anterior. Um erro grave e recorrente é usar este capítulo para REPETIR coisas que já foram ditas. Não precisa dizer aqui o que você fez, isto já foi dito no capítulo de desenvolvimento projeto. Não precisa aqui repetir todos os resultados, isto já foi apresentado antes também. Você pode referenciar os seus principais resultados para fundamentar suas conclusões.

% Outro ponto é ter cuidado com o que se conclui. Pergunte sempre se realmente seus resultados permitem concluir o que você escreveu que está concluindo. Neste ponto das conclusões é onde você deve deixar evidente qual a principal contribuição de seu trabalho e quais as contribuições complementares/secundárias. Esta é uma pergunta que todo membro de banca examinadora tem na cabeça, então se já tiver claro aqui na conclusão é melhor.

% Finalmente sugira trabalhos futuros que façam sentido a partir dos seus resultados. Trabalhos estes que poderão ser desenvolvidos por você mesmo ou outras pessoas. São uma espécie de continuidade deste trabalho.

% Depois da conclusão virão os elementos pós-textuais: Glossários, Referências Bibliográficas, Apêndices e Anexos. Apenas as Referências Bibliográficas são obrigatórias e devem estar de acordo com as normas da ABNT.

%  O Apêndice é algum conteúdo complementar não obrigatório gerado pelo próprio autor. Por exemplo, um manual de uso de um sistema, diagramas de classe de um sistema, código-fonte completo ou parcial de um sistema. Enfim, coisas que não são necessárias para entender a solução e os resultados mas que podem ser úteis para alguém que queira se aprofundar mais. Como dito, é opcional e você pode ter um ou mais apêndices.

% Anexos também são opcionais. São conteúdos complementares de outros autores. Pode ser anexado algum capítulo de outro trabalho, ou outro texto ou material relevante que não seja de sua autoria. Obviamente com a fonte devidamente referenciada.

