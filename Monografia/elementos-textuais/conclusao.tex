\chapter{Considerações finais}

Em síntese, este estudo alcançou plenamente todos os objetivos propostos, abarcando tanto as metas específicas quanto a meta geral, resultando em avanços significativos no campo da análise da evolução molecular viral.

O objetivo geral de desenvolver um novo método para a análise da evolução molecular viral foi completamente atingido, culminando na concepção e implementação bem-sucedida do \gls{agua}, uma ferramenta de análise de genes virais com base no uso de códons para classificação não supervisionada.

Quanto aos objetivos específicos, é possível destacar as seguintes conquistas:

\begin{enumerate}[label=~(\roman*)]
  \item Montagem do Dataset: Realizamos a compilação de um dataset abrangente, fundamental para as análises subsequentes.
  \item Definição de um Modelo para Validação: Estabelecemos um modelo robusto que serviu como base para a validação do método proposto, assegurando a solidez das avaliações.
  \item Desenvolvimento de uma Ferramenta para Caracterização/Validação: Criamos a ferramenta \gls{agua}, uma implementação eficaz que permitiu não apenas caracterizar, mas também validar o método proposto de forma precisa e eficiente.
  \item Coleta de Dados Necessários para a Validação: Executamos a coleta e dos dados essenciais para validar o método, garantindo que as análises fossem realizadas com informações relevantes e representativas.
  \item Comparação de Performance: Realizamos uma comparação abrangente da performance computacional do \gls{agua} com um método do estado da arte (IQ-TREE).
\end{enumerate}

Portanto, os resultados obtidos e os objetivos integralmente alcançados reforçam a contribuição significativa deste trabalho para o avanço da compreensão da evolução molecular viral, oferecendo uma abordagem inovadora por meio do \gls{agua}. Esses êxitos fortalecem a relevância do estudo no contexto científico e apontam para possíveis desenvolvimentos futuros na área.

\chapter{Trabalhos Futuros}

Este estudo abre perspectivas para investigações futuras, destacando algumas possibilidades de continuidade:

\begin{enumerate}
  \item Expansão para Outras Métricas e Espécies Virais:
        \begin{itemize}
          \item Realizar experimentos adicionais com o AGUA, explorando a sua aplicabilidade em métricas alternativas e em sequências de outras espécies virais, além do SARS-CoV-2.
          \item Avaliar o desempenho do AGUA em relação a sequências de nucleotídeos de diferentes comprimentos, analisando sua adaptabilidade a variabilidades no tamanho dos dados.
        \end{itemize}
  \item Análise Qualitativa da Eficiência do Modelo:
        \begin{itemize}
          \item Investigar a eficácia do AGUA na análise qualitativa, particularmente em sua capacidade de identificar novas espécies.
          \item Avaliar o desempenho do modelo em cenários de descoberta de espécies não previamente catalogadas, fornecendo novas visões sobre sua utilidade em contextos de pesquisa emergentes.
        \end{itemize}
  \item Desenvolvimento de uma Interface (\textit{Frontend}) para o \gls{agua}:
        \begin{itemize}
          \item Desenvolver um site que permita a fácil interação com o AGUA, tornando suas funcionalidades mais acessíveis a uma variedade de usuários, incluindo pesquisadores e profissionais da área.
        \end{itemize}
\end{enumerate}
% Neste capítulo são apresentadas duas coisas: Conclusões e Trabalhos Futuros. Aqui devem ser apresentadas a que conclusões se pode chegar a partir dos resultados apresentados e analisados no capítulo anterior. Um erro grave e recorrente é usar este capítulo para REPETIR coisas que já foram ditas. Não precisa dizer aqui o que você fez, isto já foi dito no capítulo de desenvolvimento projeto. Não precisa aqui repetir todos os resultados, isto já foi apresentado antes também. Você pode referenciar os seus principais resultados para fundamentar suas conclusões.

% Outro ponto é ter cuidado com o que se conclui. Pergunte sempre se realmente seus resultados permitem concluir o que você escreveu que está concluindo. Neste ponto das conclusões é onde você deve deixar evidente qual a principal contribuição de seu trabalho e quais as contribuições complementares/secundárias. Esta é uma pergunta que todo membro de banca examinadora tem na cabeça, então se já tiver claro aqui na conclusão é melhor.

% Finalmente sugira trabalhos futuros que façam sentido a partir dos seus resultados. Trabalhos estes que poderão ser desenvolvidos por você mesmo ou outras pessoas. São uma espécie de continuidade deste trabalho.

% Depois da conclusão virão os elementos pós-textuais: Glossários, Referências Bibliográficas, Apêndices e Anexos. Apenas as Referências Bibliográficas são obrigatórias e devem estar de acordo com as normas da ABNT.

%  O Apêndice é algum conteúdo complementar não obrigatório gerado pelo próprio autor. Por exemplo, um manual de uso de um sistema, diagramas de classe de um sistema, código-fonte completo ou parcial de um sistema. Enfim, coisas que não são necessárias para entender a solução e os resultados mas que podem ser úteis para alguém que queira se aprofundar mais. Como dito, é opcional e você pode ter um ou mais apêndices.

% Anexos também são opcionais. São conteúdos complementares de outros autores. Pode ser anexado algum capítulo de outro trabalho, ou outro texto ou material relevante que não seja de sua autoria. Obviamente com a fonte devidamente referenciada.