\chapter{Considerações finais}

Nesta fase inicial do projeto, foi realizada com sucesso a montagem do dataset de genes virais a serem estudados. Através do uso de scripts e procedimentos adequados, foram baixadas sequências classificadas de genoma completo do vírus de uma base de dados pública, como o \gls{bvbrc}, e em seguida, filtradas para manter apenas as sequências únicas. Além disso, o gene de interesse foi extraído utilizando a técnica de blast.

A montagem do dataset é uma etapa crucial para o desenvolvimento da ferramenta de análise de genes virais baseada em códons. Ao obter uma base de dados representativa e de qualidade, garantimos que a análise subsequente seja feita em um conjunto abrangente de sequências relevantes.

No entanto, é importante ressaltar que essa é apenas uma etapa inicial do projeto e que há muito trabalho a ser feito nas próximas fases. A análise comparativa entre o método proposto e outro método existente, bem como a implementação do modelo de classificação e as etapas subsequentes do processo, ainda estão por vir.

Com base na conclusão desta etapa, temos uma base sólida de dados que nos permitirá avançar no desenvolvimento do projeto. O próximo passo será a implementação do modelo de análise de genes virais baseado em codons e a realização das etapas subsequentes, como o alinhamento, a tradução e a extração de códigos únicos para a classificação e agrupamento das sequências.

A montagem bem-sucedida do dataset é um marco importante para o progresso do projeto, fornecendo a base necessária para a etapa seguinte. Com um dataset representativo e de qualidade, podemos agora prosseguir para a implementação do modelo e a análise comparativa com outras abordagens existentes.

% Neste capítulo são apresentadas duas coisas: Conclusões e Trabalhos Futuros. Aqui devem ser apresentadas a que conclusões se pode chegar a partir dos resultados apresentados e analisados no capítulo anterior. Um erro grave e recorrente é usar este capítulo para REPETIR coisas que já foram ditas. Não precisa dizer aqui o que você fez, isto já foi dito no capítulo de desenvolvimento projeto. Não precisa aqui repetir todos os resultados, isto já foi apresentado antes também. Você pode referenciar os seus principais resultados para fundamentar suas conclusões.

% Outro ponto é ter cuidado com o que se conclui. Pergunte sempre se realmente seus resultados permitem concluir o que você escreveu que está concluindo. Neste ponto das conclusões é onde você deve deixar evidente qual a principal contribuição de seu trabalho e quais as contribuições complementares/secundárias. Esta é uma pergunta que todo membro de banca examinadora tem na cabeça, então se já tiver claro aqui na conclusão é melhor.

% Finalmente sugira trabalhos futuros que façam sentido a partir dos seus resultados. Trabalhos estes que poderão ser desenvolvidos por você mesmo ou outras pessoas. São uma espécie de continuidade deste trabalho.

% Depois da conclusão virão os elementos pós-textuais: Glossários, Referências Bibliográficas, Apêndices e Anexos. Apenas as Referências Bibliográficas são obrigatórias e devem estar de acordo com as normas da ABNT.

%  O Apêndice é algum conteúdo complementar não obrigatório gerado pelo próprio autor. Por exemplo, um manual de uso de um sistema, diagramas de classe de um sistema, código-fonte completo ou parcial de um sistema. Enfim, coisas que não são necessárias para entender a solução e os resultados mas que podem ser úteis para alguém que queira se aprofundar mais. Como dito, é opcional e você pode ter um ou mais apêndices.

% Anexos também são opcionais. São conteúdos complementares de outros autores. Pode ser anexado algum capítulo de outro trabalho, ou outro texto ou material relevante que não seja de sua autoria. Obviamente com a fonte devidamente referenciada.

