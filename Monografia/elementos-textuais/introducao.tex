%%%%%%%%%%%%%%%%%%%%%%%%%%%%%%%%%%%%%%%%%%%%%%%%%%%%%%%%%%%%%%%%%%%%%%%%%%%
%%%                         INTRODUÇÃO                                  %%%
%%%%%%%%%%%%%%%%%%%%%%%%%%%%%%%%%%%%%%%%%%%%%%%%%%%%%%%%%%%%%%%%%%%%%%%%%%%

\chapter{Introdução}

\setlength{\parskip}{0.3cm}
%\thispagestyle{headings}

% Contextualização, relevância (importância, justificativa, necessidade) do problema
Os vírus causam doenças de vários níveis de gravidade nos humanos, plantas e animais, sendo de suma importância a sua descoberta e caracterização, para entender e controlar as enfermidades e atender aos graus de sofrimento econômico e individual por elas causados. Além disso, o seu estudo fornece de forma continua a base para grande parte da compreensão mais fundamental da biologia, genética e medicina modernas~\cite{virology_edward_2007}.

Dessa forma, é indubitável que os problemas impostos pela pandemia do novo coronavírus (do inglês \gls{covid19}) incluíram a falta de entendimento suficiente para a compreensão da importância das ameaças biológicas, além da inexperiência médica para lidar com o novo impasse sanitário que surgiu, apesar dos avanços científicos e tecnológicos já alcançados na área em questão~\cite{coronavirus_binsfeld_2020}. Em vista disso, o conhecimento prévio sobre os agentes nocivos com potencial para causar pandemias, tem o poder de melhorar substancialmente uma preparação pré-pandemia~\cite{behl_threat_2022}.

Dentre esses desafios, o \gls{sarscov2} emergiu como o patógeno de maior prioridade entre 2020 e 2023, devido à sua alta transmissibilidade e a ausência de medidas eficazes. A limitação do conhecimento sobre a epidemiologia, imunidade e patogênese do vírus destacou a necessidade de aprimorar a biossegurança e expandir a capacidade dos laboratórios e profissionais de saúde~\cite{sarscov2_covidsurg_2021,sarscov2_deigin_2021, multiple_kuchipudi_2022, efficacy_madhi_2021}. Com mais de 771 milhões de casos confirmados e quase 7 milhões de mortes globalmente, sendo mais de 193 milhões de casos e quase 3 milhões de mortes concentrados nas Américas~\cite{who_covid19}, ações intergovernamentais foram estabelecidas para responder à pandemia e controlar a transmissão do vírus.

Diante desse cenário, a bioinformática, que é a junção de métodos computacionais e técnicas estatísticas que possuem o objetivo de extrair informações de dados biológicos brutos, desempenha um papel basilar na interpretação de dados genômicos e na compreensão de processos evolutivos~\cite{herramientas_gonzales_2016}. Ademais, é o conjunto com a disciplina de análise genética que é possível compreender de forma crucial a diversidade e evolução de vírus que podem afetar organismos, incluindo seres humanos, animais e plantas. Essa investigação não apenas fornece \textit{insights} sobre a classificação e identificação de patógenos, mas também é valiosa para a pesquisa em saúde pública, agronegócio e ecologia. No entanto, a complexidade inerente às sequências genéticas e a crescente disponibilidade de dados desafiam a capacidade de diagnóstico humana~\cite{virology_edward_2007, diversidade_flores_2000}.

A reconstrução filogenética é uma das abordagens amplamente utilizadas no estudo da evolução de espécies, a qual permite investigar as relações evolutivas entre diferentes linhagens de vírus~\cite{sistematica_santos_2012, consideracoes_ribas_2006,dinossauros_santos_2008}. Essas observações são realizadas com base nas sequências de \gls{dna} e \gls{rna}, as quais são formadas respectivamente por blocos fundamentais chamados de nucleotídeos, que são compostos por uma base nitrogenada, um açúcar e um grupo fosfato. Por conseguinte as bases presentes nos nucleotídeos do \gls{dna} são \gls{adenina}, \gls{timina}, \gls{citosina} e \gls{guanina}, enquanto no \gls{rna} a base \gls{timina} é substituída pela \gls{uracila}~\cite{alberts_molecular_2002,molecular_bernard_2022,genetica_peter_2017}.
Segundo~\citeauthoronline{barry_phylogenetic_analysis_2006}, os métodos filogenéticos podem ser usados para analisar os dados da sequência de nucleotídeos de forma que a ordem de descendência de cepas relacionadas possa ser determinada. Portanto, quando associada à análise filogenética apropriada, a epidemiologia molecular tem o potencial de elucidar os mecanismos que levam a surtos microbianos e epidemias.

Uma das principais formas de apuração é realizada através da árvore filogenética, onde são representadas as relações evolutivas entre um conjunto de espécies. De acordo com~\citeauthoronline{morrison_tree_thinking}, elas tem função importante porque apresentam de forma sucinta e particular a evolução dos descendentes partindo de ancestrais em comum.

% Justificativas e Contribuições
Seguindo a linha dos métodos até então desenvolvidos, este trabalho busca tentar desenvolver um método de construção de árvores com base nas distâncias obtidas a partir da diferença do uso de códons, e assim poder contribuir com a tarefa de classificação de cepas para entes responsáveis por controles voltados à área da saúde, especialmente na descoberta daquelas cepas emergentes com potenciais pandêmicos. Ademais, é também importante dispor de alternativas à filogenia molecular atualmente utilizada, para gerar informações de outro ponto de vista e-ou para servir de referência aos métodos filogenéticos~\cite{virology_flint_2015}.

Visto que os processos atuais ainda demandam de um alto custo computacional, devido principalmente a quantidade de dados a serem tratados, é notória a necessidade de desenvolver outros mais baratos e que possam suportar o volume crescente de dados (sequências).
% Apresentação da Nossa Abordagem
Sendo assim, o projeto visa apresentar um método que seja capaz de realizar classificações, com um custo computacional baixo, em relação a outros, e que possa apresentar, do ponto de vista científico, alternativas de comparação com outras técnicas já existentes~\cite{frank_chemistry_2017,gene_brow_2020}. A hipótese referente à menor complexidade computacional da nova metodologia deverá ser testada no trabalho.
% Fim Justificativa

A semelhança genética entre vários vírus infecciosos e mortais fornece uma visão do fato de que o \gls{rna} é a chave para discernir e marcar os possíveis patógenos que podem causar uma pandemia. Embora um padrão geral e motivos conservados possam ser observados em ancestrais imediatos, as regiões não conservadas das sequências são o resultado da acumulação de mutações, seja por inserção ou deleção de um ou vários nucleotídeos ou por substituição pontual de um nucleotídeo por outro. A fonte principal de mutações em vírus são percalços na replicação e a recombinação de \gls{rna}~\cite{behl_threat_2022}.

Apesar da utilidade da filogenética e dos softwares comerciais e públicos disponíveis para análises, os métodos propostos por ela são muitas vezes aplicados de forma inadequada. A aplicação inadequada de métodos filogenéticos pode levar a resultados imprecisos e interpretações incorretas. Algumas das práticas inadequadas podem incluir a escolha inadequada de modelos evolutivos, a falta de avaliação estatística adequada, a utilização de dados de baixa qualidade ou a interpretação incorreta dos resultados filogenéticos~\cite{felsenstein_inferring_2004,mrbayes_huelsenbeck_2001}. Mesmo quando operados adequadamente, são mal explicados e, portanto, mal compreendidos~\cite[p. 1]{barry_phylogenetic_analysis_2006} Além disso, por trabalhar com grandes quantidades de dados, as técnicas utilizadas devem ser avaliadas também em relação ao seu custo computacional.

% Problema de Pesquisa
% É possível realizar a classificação de sequências genéticas com base no uso de códons
% Repetição encontrada acima na parte de apresentação da nossa abordagem
Até o momento, as soluções desenvolvidas, têm sido direcionadas pela reconstrução de árvores filogenéticas, as quais são construídas a partir de diferentes informações genéticas. Uma abordagem comum utiliza as mutações de nucleotídeos em sequências de \gls{dna}. Além disso, há métodos que exploram a evolução de sequências de aminoácidos em proteínas, constituindo a filogenia baseada em aminoácidos~\cite{protein_le_2008}. Outra perspectiva relevante é a filogenia baseada em distâncias, que calcula as distâncias evolutivas entre sequências, podendo utilizar métricas como a distância de Jaccard ou a distância de Hamming~\cite{sokal_statistical_method_1958}. Essas diferentes estratégias contribuem para uma compreensão mais abrangente das relações evolutivas entre organismos. Neste aspecto, as ferramentas disponíveis não oferecem uma aplicação no contexto de árvores reconstruídas com distâncias obtidas a partir da diferença do uso de códons. Estes são sequências de três nucleotídeos responsáveis pela codificação dos aminoácidos nas proteínas. Os códons desempenham um papel crucial na determinação da função e estrutura das proteínas, e alterações nos códons podem resultar em mudanças significativas nas características fenotípicas dos vírus. Portanto, é necessário a realização de pesquisas e desenvolvimento de instrumentos que sejam capazes de classificar sequências genéticas com base no uso de códons.

% Objetivos do Anteprojeto
Com base no problema de pesquisa proposto, foram construídos os objetivos, os mesmos são apresentados a seguir:
\begin{itemize}
      \item \textbf{Objetivo Geral}
            \begin{enumerate}[label=~(\roman*)]
                  \item Desenvolver um novo método de análise da evolução molecular viral baseado no uso de códons.
            \end{enumerate}
      \item \textbf{Objetivos Específicos}
            \begin{enumerate}[label=~(\roman*)]
                  \item Montar dataset do projeto
                  \item Definir um plano de implementação do método proposto
                  \item Desenvolver uma de ferramenta para caracterizar/validar o método
                  \item Coletar os dados necessários para validar o método
                  \item Realizar a comparação da performance computacional do novo método com algum dos métodos do estado da arte.
                        % \item Disponibilizar o modelo como uma ferramenta web de fácil acesso.
            \end{enumerate}
\end{itemize}

% Estrutura da monografia
Esta monografia possui uma estrutura cuidadosamente elaborada para abordar de forma abrangente o projeto de desenvolvimento da ferramenta de análise de genes virais baseada no uso de códons. A seguir, é apresentada cada seção, delineando seu conteúdo e importância na apresentação do trabalho:

\begin{itemize}
      \item Capítulo 1: Introdução

            O capítulo introdutório contextualizará o problema abordado, destacando sua relevância, importância e necessidade na área de análise de genes virais. Além disso, apresentará a estrutura da monografia, fornecendo uma visão geral das seções subsequentes.

      \item Capítulo 2: Fundamentação Teórica

            Este capítulo estabelecerá as bases teóricas para o projeto. Explora conceitos essenciais relacionados a genes virais, códons, filogenética, metodologia DSR e outras áreas relevantes. Ele é fundamental para a compreensão dos métodos e resultados apresentados posteriormente.

      \item Capítulo 3: Descrição do Projeto

            Neste capítulo, será detalhado o projeto em sua totalidade. Isso inclui a metodologia utilizada, materiais e métodos empregados, bem como o plano de implementação. Também, abordará a montagem e preparação do dataset, desenvolvimento do modelo e a análise comparativa com outro existente.

      \item Capítulo 4: Montagem e Preparação do Dataset

            Este capítulo destaca o processo de montagem do dataset, descrevendo todos os seus procedimentos:\textit{download}, filtragem, alinhamento, extração de genes de interesse e a remoção de sequências duplicadas. Essas etapas são fundamentais para obter um dataset de alta qualidade.

      \item Capítulo 5: Desenvolvimento do Modelo

            Este capítulo, irá detalhar a implementação do modelo de classificação não supervisionada com base em códons. Isso envolverá a tradução de sequências de DNA, a extração de códigos únicos e o processo de agrupamento. Além disso, o capitulo abordará a associação de clusters com classes de sequências.

      \item Capítulo 6: Análise Comparativa do Modelo Proposto e Outro Existente

            Nesta seção, será realizada uma análise comparativa entre o método desenvolvido neste projeto e as técnicas clássicas de filogenética. Além disso, ocorrerá a avaliação de desempenho, precisão e eficiência do nosso modelo em relação a um tradicional.

      \item Capítulo 7: Considerações Finais

            A última seção da monografia destacará as principais conclusões, contribuições do projeto e recomendações para pesquisas futuras. Também enfocará as implicações práticas da ferramenta desenvolvida e seu impacto na análise de genes virais.
\end{itemize}

% Apresentação da minha abordagem/Resultados esperados
% Principais diferenciais