%%%%%%%%%%%%%%%%%%%%%%%%%%%%%%%%%%%%%%%%%%%%%%%%%%%%%%%%%%%%%%%%%%%%%%%%%%%
%%%                         O MODELO                                    %%%
%%%%%%%%%%%%%%%%%%%%%%%%%%%%%%%%%%%%%%%%%%%%%%%%%%%%%%%%%%%%%%%%%%%%%%%%%%%

\chapter{Descrição do Projeto}

% objetivos, ponto de partida, produtos e resultados esperados, tecnologias a serem usadas.
% + estrutura da solução nas fases de treinamento e aplicação

% ### METODOLOGIA ###

% ### Falar das seguintes ferramentas:
%   - DSR
%   - Trello
%   - Github

A seguir, serão apresentadas a metodologia e os softwares utilizados neste estudo, bem como as etapas detalhadas de sua implementação.

% Metodologia
\section{Metodologia}~\label{sec:metodologia}
Um ponto importante para a obtenção dos objetivos deste trabalho está relacionada a definição da metodologia que servirá como alicerce. Com a proposta de desenvolver e validar um método de análise da evolução molecular de vírus com base no uso de códons, a metodologia escolhida para isso é o \gls{dsr}. Essa metodologia, proporciona um framework teórico e prático para a criação de artefatos inovadores, como métodos, modelos ou frameworks, visando resolver problemas específicos~\cite{peffers_dsr_2007}. Neste projeto, a ferramenta de análise de genes virais baseada em códons é o artefato que será desenvolvido e avaliado. Além disso, o \gls{dsr} enfatiza a validação e a avaliação da utilidade e eficácia do artefato em relação aos seus objetivos práticos. No caso deste projeto, a validação será realizada através da comparação dos resultados obtidos com a ferramenta proposta em relação às técnicas clássicas filogenéticas, que são amplamente utilizadas para a análise de genes virais. Essa comparação permitirá avaliar a eficácia e o valor agregado da abordagem baseada em códons.

Para a obtenção de sucesso ao utilizar o \gls{dsr} os seguintes passos serão seguidos:
\begin{enumerate}
  \item Identificação do problema e definição dos objetivos.
  \item Desenvolvimento dos artefatos.
  \item Avaliação do artefato.
  \item Apresentar contribuições científicas.
\end{enumerate}

Também será utilizada analises quantitativas, ou seja, medidas estatísticas para mensurar e comparar os resultados obtidos.\\
A pesquisa quantitativa só tem sentido quando há um problema muito bem definido e há informação e teoria a respeito do objeto de conhecimento, entendido aqui como o foco da pesquisa e/ou aquilo que se quer estudar. Esclarecendo mais, só se faz pesquisa de natureza quantitativa quando se conhece as qualidades e se tem controle do que se vai pesquisar.~\cite{da_silva_pesquisa_2014}

\section{Materiais e Métodos}
Nesta sessão, será apresentada as ferramentas utilizadas para a construção e desenvolvimento de todo o trabalho.

O Python é uma linguagem de programação de alto nível, interpretada, iterativa e de código aberto. Foi criada por Guido van Rossum e lançada em 1991. A linguagem é conhecida por ter uma sintaxe simples, tornando-a popular para o desenvolvimento de software, automação, análise de dados, aprendizado de máquina entre outras aplicações. A mesma apresenta suporte a vários paradigmas de programação, como a orientada a objetos, imperativa, procedural e funcional. Além disso, o Python é portátil, podendo ser executado em diversos sistemas operacionais como linux, mac e windows.~\cite{python-reference}

Para a construção dos pipelines do projeto, utilizamos Python em conjunto com o Jupyter Notebook. O Jupyter Notebook é uma aplicação de código aberto que permite criar documentos interativos que integram código, texto narrativo e visualizações. É uma ferramenta amplamente adotada por cientistas de dados, pesquisadores e desenvolvedores para explorar dados, prototipar código, documentar projetos e facilitar a colaboração. Além disso, o Jupyter Notebook oferece suporte a diversas linguagens de programação, incluindo Python~\cite{jupyter-notebook}.

O python possue uma gama de bibliotecas que facilitam a implementação de soluções complexas. A seguir serão apresentadas a bibliotecas utilizadas:

\begin{itemize}
  \item \textbf{Biopython}: Coleção de bibliotecas e ferramentas em Python, disponíveis gratuitamente para biologia molecular computacional. Ele fornece uma ampla gama de funcionalidades, desde a leitura e análise de arquivos de sequência biológica até a execução de algoritmos sofisticados de bioinformática. Desenvolvida e mantida pelo Projeto Biopython, que é uma associação internacional de desenvolvedores de ferramentas python.~\cite{biopython}
  \item \textbf{Selenium}: Biblioteca de código aberto que fornece uma interface programática para automatizar interações com navegadores da web. É amplamente utilizado por desenvolvedores e testadores de software para realizar testes automatizados, raspagem de dados na web e outras tarefas que envolvem interações com páginas da web. O Selenium para Python permite a automação de ações como clicar em botões, preencher formulários, navegar em sites e extrair informações da web, tornando-o uma ferramenta valiosa para desenvolvimento e automação de tarefas na web.~\cite{selenium-python}
\end{itemize}

% Lista de Tecnologias/Ferramentas utilizadas:
% Python 3 ok
% Selenium ok
% Biopython ok
% Shell (verificar necessidade)
% Jupyter Notebook ok
% Minimap2
% GoFasta
% virtualEnv
% BV_BRC - O \gls{bvbrc}, é um centro de recursos de bioinformática dedicado ao estudo e análise de bactérias e vírus. O site também disponibiliza uma uma coleção abrangente de banco de dados, incluindo sequências genômicas, anotações funcionais, informações de expressão gênica e estruturas tridimensionais. O acesso aos bancos de dados é dado por meio de uma interface amigável, onde é possível realizar pesquisas avançadas.

% AGUA (verificar e apresentar os sub-softwares utilizados, ex.: CLOPE)

\section{Plano de Implementação}
% Lista de Tecnologias/Ferramentas utilizadas:
% Python 3
% Selenium
% Biopython
% Shell (verificar necessidade)
% Jupyter Notebook
% Minimap2
% GoFasta
% virtualEnv
% AGUA (verificar e apresentar os sub-softwares utilizados, ex.: CLOPE)

Durante o desenvolvimento do projeto foi necessário dividir o projeto em fases com base nas atividades que deveriam ser realizadas de forma a atender todos os passos descritos na seção~\ref{sec:metodologia}. As principais fases identificadas foram: Montagem e preparação do dataset a ser utilizado pelo modelo; Desenvolvimento completo do modelo, com todos as definições, implementações, testes e correções necessárias; e a análise comparativa que será realizada com um outro método existente e já tradicional. Esses pontos são apresentados de forma minuciosa a seguir.

\subsection{Montagem e Preparação do Dataset}
% Verificar como descrever os dois conjuntos de pipelines montados. Pip 01 (Descartado) - Pip 02 (Atual)
Para realizar o treinamento do modelo a ser construído, eram necessárias sequências únicas e alinhadas do gene Spike. Em vista disso, é importante salientar que o site \gls{bvbrc} disponibiliza sequências genômicas, e sendo assim, foi preciso construir um pipeline para, após o download das sequências, transformar as mesmas para a criação de um dataset com as sequências que atendessem os requisitos esperados.

Inicialmente, foi realizada uma analise do \gls{bvbrc}, para entende a sua estrutura e verificar também se era possível realizar o download de todas as sequências queridas de forma manual. Foi verificado que o site possuía uma área de seleção de filtros, e foi definido que só seriam selecionadas sequências completas no campo \textit{Genome Status} e no campo \textit{Lineage}, onde é possível filtrar as sequências pelo seu tipo \textit{Pango} e também verificar a quantidade, só os que tivessem mais de 50 sequências do mesmo tipo.

Após a análise, foi constatado que realizar o download manualmente era infactível, e que seria preciso automatizar esse processo de iteração com a página, como vistos na Figura~\ref{fig:pipelineBvbrc}. Isto posto, foi realizado uma sequência de passos conhecidos como \textit{Web Scrapping} utilizado o Python juntamente com o Selenium, apresentados em seguida:

\begin{enumerate}
  \item teste 01
  \item teste 02
\end{enumerate}

Ao final do processo de montagem do dataset com sequências genômicas completas, o mesmo ficou com as seguintes informações apresentadas na tabela x.

\begin{figure}[htb]
  \centering
  \caption{Pipeline de Download das Sequências Genômicas.}
  \includegraphics[scale=0.6]{figuras/pipelines/bvbrc.png}
  \fonte{O Autor}~\label{fig:pipelineBvbrc}
\end{figure}

A montagem do dataset se deu com a construção de pipelines de processamento das sequências. Um pipeline inicial foi construído e será apresentado a seguir, o mesmo foi descontinuado após se verificar que o tempo de processamento ficaria inviável.

\begin{itemize}
  % Verificar se é melhor descrever o processo ou listar os itens como está atualmente
  \item \textbf{Montagem e Preparação do Dataset:}  Abaixo, está apresentado o passo a passo que foi realizado nesta etapa:
        \begin{itemize}
          \item Analise do site \gls{bvbrc}.
          \item Desenvolvimento de um script Python juntamente com o Selenium, para o download automático das sequências genomicas no \gls{bvbrc}.
          \item Filtragem das sequências genômicas duplicadas, ou seja, que contenham exatamente, a mesma sequência de nucleotídeos, mantendo apenas 1(uma) das repetidas e construindo um dataset de sequências genomicas únicas. (Filtro 01)
          \item Alinhamento das sequências genomicas utilizando o Minimap2.
          \item Implementado procedimento de extração do gene de interesse (Spike), com base em uma sequência Spike de referência, das sequências únicas.
          \item Filtragem das sequências genicas duplicadas, ou seja, que contenham exatamente, a mesma sequência de nucleotídeos, mantendo apenas 1(uma) das repetidas e construindo um dataset de sequências genicas únicas. (Filtro 02)
          \item Filtragem das sequências genicas de má qualidade. Foram removidas as sequências que possuiam mais de 30 N's consecultivos ou que ficaram com a quantidade de nucleotídeos discrepantes em relação a referência genica. (Filtro 03)
          \item Criar arquivo de treinamento com linhagens diferentes e um arquivo de anotação. % Descrever melhor os arquivos
        \end{itemize}


  \item \textbf{Desenvolvimento do Modelo:} Para a realização do desenvolvimento do modelo serão realizados os seguintes passos:
        % FAZER
        % Escrever uma descrição melhor do desenvolvimento do modelo
        \begin{itemize}
          \item Levantamento dos requisitos.
          \item Definir a arquitetura e a abordagem do modelo de classificação baseado em códons.
          \item Implementar o modelo utilizando uma biblioteca ou framework adequado.
          \item Desenvolver algoritmo para traduzir as sequências de \gls{dna} em sequências de códons.
          \item Realizar treinamento do modelo utilizando os dados preparados.
          \item Avaliar o desempenho do modelo utilizando métricas apropriadas.
                % \item \item Avaliar o desempenho do modelo utilizando métricas apropriadas, como acurácia, precisão e recall.
          \item Identificar possíveis problemas e realizar ajustes no modelo.
                % \item Identificar possíveis problemas de overfitting ou underfitting e realizar ajustes no modelo, como ajuste de hiperparâmetros ou utilização de técnicas de regularização.
        \end{itemize}

  \item \textbf{Análise comparativa entre o método proposto e outro método existente:} Será realizada uma analise com um conjunto de dados, onde será realizado analises estatísticas para verificação de melhorias, ou não, do novo método proposto analisando os seguintes aspectos:
        \begin{itemize}
          \item Comparação dos métodos de agrupamento adotados, avaliando sua eficácia na formação de clusters e na identificação de padrões ou similaridades nas sequências.
          \item Avaliação do custo computacional (tempo de execução e recursos requeridos) para a classificação das sequências em cada método.
          \item Comparação da eficiência computacional entre os métodos, considerando a escalabilidade e o desempenho em grandes volumes de dados.
        \end{itemize}
\end{itemize}

% Depois dos capítulos de fundamentação teórica e trabalhos relacionados, é hora de apresentar o trabalho propriamente dito. Descrever o que foi feito para desenvolver a solução do problema investigado. Aqui a metodologia deve ser bem detalhada, a arquitetura das soluções descritas em detalhes, as tecnologias utilizadas apresentadas e justificadas. Enfim, aqui é uma espécie de relatório do que você fez.

% Mais uma vez, alguns autores preferem neste mesmo capítulo apresentar também o plano de testes e validação e os resultados obtidos. Outros autores colocam isto num capítulo separado depois deste. Novamente não há regra rígida para isto. As duas opções são válidas e devem ser decididas em conjunto com o orientador. Sempre usar o bom senso para não ter capítulos curtos demais nem longos demais. É importante deixar muito claro o plano de validação, a fundamentação estatística utilizada, o processo de coleta de dados, e apresentar o resultado de forma adequada usando gráficos, tabelas, etc. Estes resultados devem ser analisados, mas NÃO DEVEM GERAR CONCLUSÕES AINDA. Analisar não significa concluir. Para isto existe um capítulo de Conclusão!


% \section{Plano de Testes e Validação}

% \section{Resultados Obtidos}