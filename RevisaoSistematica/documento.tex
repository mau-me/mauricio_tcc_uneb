\documentclass[12pt]{article}


\usepackage{sbc-template}

\usepackage{graphicx,url}

\usepackage{amsmath}

\usepackage[colorinlistoftodos]{todonotes}

\usepackage[english,brazil]{babel}   
%\usepackage[latin1]{inputenc}  
\usepackage[utf8]{inputenc}  
\usepackage[T1]{fontenc}
% UTF-8 encoding is recommended by ShareLaTex
\usepackage{multirow}
\usepackage{longtable}
\usepackage{listings}
\usepackage{xcolor}
\usepackage{pdflscape}
\usepackage{bibentry}
\nobibliography*

\renewcommand{\lstlistingname}{Código}
\DeclareUnicodeCharacter{00A0}{~}



\setlength{\LTcapwidth}{\linewidth}

\sloppy

\title{Uma Revisão de Literatura sobre \textit{Ferramentas Computacionais para
        o Estudo da Evolução de Espécies Baseado no Uso de Códons}}

\author{Mauricio Souza Menezes\\}
\address{Departamento de Ciências Exatas e da Terra, Campus I\\
    Universidade do Estado da Bahia (UNEB)\\
    Salvador, Bahia, Brasil.
    \email{mauriciosm95@gmail.com}
}

\begin{document}

\newacronym{ufm}{UFM}{Universal Feature Method}
\newacronym{ngs}{NGS}{Sequenciamento de Nova Geração}
\newacronym{rs}{RS}{Revisão Sistemática}
\newacronym{cafe}{CAFe}{Comunidade Acadêmica Federada}
\newacronym{parsifal}{Parsifal}{Perform Systematic Literature Reviews}
\newacronym{fcgr}{FCGR}{Frequency Chaos Game Representation}

\maketitle

\begin{resumo}
    Esta \gls{rs} teve como objetivo buscar a sustentação relacionada a genotipagem de sequências, levantando quais métodos são utilizados.
    Vários métodos diferentes foram identificados, demostrando assim, que nenhum deles foi considerado como indubitável.
\end{resumo}

\begin{abstract}
    \begin{otherlanguage}{english}
        This Systematic Review (SR) aimed to seek support related to sequence genotyping, identifying which methods are used. Several different methods were identified, thus demonstrating that none of them was considered indubitable.
    \end{otherlanguage}
\end{abstract}

\section{Introdução}

A análise filogenética é uma ferramenta de extrema importância para a virologia. Sua avaliação é parte essencial para definição, por exemplo, da evolução de agentes causadores de doenças.
Segundo~\cite{behl_threat_2022} `existe uma tendência histórica, onde os vírus e bactérias sofrem mutações no decorrer do tempo, encontrando mecanismos para infectar as células humanas.'
Este artigo apresenta uma \gls{rs} da literatura referente aos métodos computacionais utilizados para realizar o estudo de especiação genômica. Os artigos foram obtidos através da busca realizada em quatro bases de dados (PubMed, ScienceDirect, Scopus e Springer Link) levando em conta a principal questão: Quais métodos são utilizados para realizar o estudo de especiação? Foram incluídos todos os artigos que atenderam aos critérios definidos.

\section{Relato da Revisão de Literatura}

Logo após definir os objetivos da pesquisa, foi primordial conhecer os pormenores que estavam relacionados aos mesmos, e para isso foi necessário realizar uma \gls{rs}. Dessa forma, foi utilizado um protocolo que serviria de base no processo para a obtenção dos estudos que deveriam ser analisados, os quais foram definidos e analisados através de: Objetivos, questões, palavras-chave, strings de busca, fontes e critérios para inclusão e exclusão.

\subsection{Objetivo da Pesquisa}

A pesquisa teve como objetivo verificar e conhecer a existência de métodos utilizados para a analise da evolução de espécies com base no uso de códons, a identificação do que já foi feito para sanar o problema identificado e também se era viável ou não o desenvolvimento de uma nova solução.

\subsection{Questões de Pesquisa}

Foram definidas duas questões de pesquisa. A primeira e principal questionava quais métodos são utilizados para realizar o estudo o estudo de especiação. Enquanto a segunda era se alguma dessas metodologias é baseada no uso/frequência de códons. As questões de pesquisa foram escolhidas com base naquilo que se queria entender ao final da \gls{rs}.

\subsection{Repositório de Busca de Dados}

As bases precisavam conter trabalhos relevantes relacionados a bioinformática, completos e gratuitos. No entanto, em alguns casos, esse acesso se deu através por meio do portal da \gls{cafe}. Deste modo, as seguintes bases de dados acadêmicos foram selecionadas:
\begin{itemize}
    \item{PubMed}
    \item{ScienceDirect}
    \item{Scopus}
    \item{Springer Link}
\end{itemize}

Para o refinamento das consultas foram utilizados os filtros que serão apresentados a seguir:

\begin{enumerate}
    \item ScienceDirect
          \begin{enumerate}
              \item Article type: Review Articles e Research Articles
              \item Access type: Open access e Open Archive
              \item Years: 2018 até 2022
          \end{enumerate}
    \item Springer Link
          \begin{enumerate}
              \item Include preview-only content: Desmarcado
              \item Content type: Article
              \item Language: English
              \item Years: 2018 até 2022
          \end{enumerate}
    \item Scopus
          \begin{enumerate}
              \item Open access: All open access
              \item Years: 2018 até 2022
          \end{enumerate}
\end{enumerate}


\subsection{Palavras-chave e Strings de Busca}

As palavras-chave, com seus sinônimos e correlatos em inglês, selecionadas para a busca nas bases de dados, com o objetivo de encontrar o máximo de estudos relevantes, foram as seguintes:
\begin{itemize}
    \item{Bioinformatics e Bioinformática}
    \item{Codon e códon}
    \item{Gene}
    \item{Phylogeny e filogenia}
    \item{Classification, classificação, genotyping, genotipagem, subtyping, subtipagem, typing e tipagem}
    \item{Viral}
\end{itemize}
A tabela~\ref{tab:stringBuscaBase} apresenta as strings de busca utilizadas em suas respectivas bases de dados.

\begin{table}[h]
    \centering

    \caption{STRINGS DE BUSCA NAS BASES DE DADOS}
    \label{tab:stringBuscaBase}
    \begin{tabular}{|p{2cm}|p{10cm}|}
        \hline
        BASE DE DADOS  & STRING BUSCA

        \\
        \hline
        PubMed         & ((`bioinformatics') AND (`codon') AND (`phylogeny') AND (`typing' OR `classification' OR `genotyping' OR `subtyping') AND (`gene') AND (`viral'))
        \\
        \hline
        Science Direct & ((`bioinformatics') AND (`codon') AND (`phylogeny') AND (`typing' OR `classification' OR `genotyping' OR `subtyping') AND (`gene') AND (`viral')) \\
        \hline
        Scopus         & ((`bioinformatics') AND (`codon') AND (`phylogeny') AND (`typing' OR `classification' OR `genotyping' OR `subtyping') AND (`gene') AND (`viral'))
        \\
        \hline
        Springer Link  & ((`bioinformatics') AND (`codon') AND (`phylogeny') AND (`typing' OR `classification' OR `genotyping' OR `subtyping') AND (`gene') AND (`viral'))
        \\
        \hline
    \end{tabular}

\end{table}

\subsection{Critérios de seleção}

Para realizar o processo de seleção dos estudos foram definidos critérios, podendo ser de inclusão ou de exclusão.
Os critérios de inclusão foram os seguintes:

\begin{itemize}
    \item I1: O estudo contém as palavras-chave definidas na pesquisa no resumo, título ou palavras-chave.
    \item I2: Aborda algum método de especiação.
\end{itemize}
Os critérios de exclusão foram os seguintes:
\begin{itemize}
    \item E1: O tema do estudo não está relacionado à especiação.
    \item E2: O tema do estudo não é pertinente com a área/objetivos da pesquisa.
    \item E3: O estudo está duplicado.
\end{itemize}

\section{Resultados Parciais}

A aplicação das strings nas bases de dados selecionadas, resultou em um total geral de 177 (cento e setenta e sete) artigos, os quais foram oriundos da fonte PubMed 3 (três) artigos, da fonte ScienceDirect 85 (oitenta e cinco) artigos, da fonte Scopus 10 (dez) artigos e da fonte Springer Link 79 (setenta e nove) artigos.\\
Após o processo de obtenção, foi iniciada a seleção dos artigos, onde foi realizada a leitura do título, resumo e palavras-chave, aplicando os critérios de inclusão e exclusão. Esse processo resultou em: 26 (vinte e seis) artigos aceitos; 147 (cento e quarenta e sete) artigos rejeitados; e 4 (quatro) artigos duplicados. Os resultados são apresentados nos gráficos nas figuras~\ref{fig:artigoBases} e \ref{fig:excluidosCriterios}.\\


\begin{figure}[tb]
    \centering
    \includegraphics[scale=0.43]{figuras/_artigosAceitosSelecionados.png}
    \caption{Artigos Selecionados e Aceitos por Base.}\label{fig:artigoBases}
\end{figure}

\begin{figure}[tb]
    \centering
    \includegraphics[scale=0.5]{figuras/_artigosExcluidosPorCriterio.png}
    \caption{Artigos Selecionados e Aceitos por Base.}\label{fig:excluidosCriterios}
\end{figure}


\subsection{Análise Qualitativa dos Resultados}

Na etapa de extração, realizada com o auxílio da ferramenta \gls{parsifal}, houve o levantamento da resposta para 4(quatro) informações importantes para o trabalho: Qual método de genotipagem foi utilizado; Se necessitava de algum treinamento, e em caso positivo, se o mesmo era supervisionado e se necessitava de uma árvore de referência supervisionada. Essas informações foram importantes também para a construção da planilha-resumo de resultados. Também é importante salientar, que após a leitura completa, 12(doze) dos trabalhos se enquadraram em um dos critérios de exclusão apresentados anteriormente.

O trabalho~\cite{dimitrov_updated_2019} apresentou e comparou três métodos para a construção de árvores filogenéticas: junção de vizinhos; máxima verossimilhança e inferência bayesiana.
Já em~\cite{yin_systematic_2019} e~\cite{bedoya-pilozo_molecular_epidemiology_2018} foi empregada a inferência bayesiana. Temos que em~\cite{fall_genetic_diversity_2021},~\cite{behl_threat_2022},~\cite{shabbir_comprehensive_2020},\cite{hudu_hepatitis_2018} e~\cite{cho_analysis_2022} foi aplicada a máxima verossimilhança.
Os demais estudos apresentaram metodologias distintas: No trabalho de~\cite{lichtblau_alignment-free_2019} foi \gls{fcgr};~\cite{kim_ngs_2022} a floresta aleatória;~\cite{sallard_tracing_2021} construiu as árvores por meio de inferências filogenéticas;~\cite{paez-espino_diversity_evolution_2019} o alinhamento concatenado;~\cite{potdar_phylogenetic_2021} a junção de vizinhos e~\cite{tang_evolutionary_2021} a máxima parcimônia.

\section{Conclusões}

A necessidade do desenvolvimento de novas ferramentas ficou evidente no decorrer de toda a pesquisa, tendo em vista que a maioria dos trabalhos não apresentavam um comparativo conclusivo entre as análises filogenéticas. Sendo assim, uma nova metodologia para genotipagem de sequências contribuiria com um grande valor científico nessa área.

\bibliographystyle{sbc}
\bibliography{RefGeral}

\pagebreak
\begin{landscape}

    \section{Planilha-resumo de Resultados}

    Na~\ref{tab:resumo} é apresentada a planilha-resumo de resultados com os trabalhos aceitos no processo de seleção. Assim, para cada artigo foi extraído a sua identificação, os critérios de inclusão ou exclusão que foram aplicados, uma breve descrição e uma avaliação qualitativa.

    \begin{center}

        \begin{longtable}{p{8cm}|c|c|c|c|c|p{7cm}|p{5cm}}
            \caption{Planilha-resumo dos trabalhos selecionados.}\label{tab:resumo}
            \\
            \multicolumn{1}{c|}{\textbf{Identificação do Trabalho}} &
            \multicolumn{1}{c|}{\textbf{I1}}                        &
            \multicolumn{1}{c|}{\textbf{I2}}                        &
            \multicolumn{1}{c|}{\textbf{E1}}                        &
            \multicolumn{1}{c|}{\textbf{E2}}                        &
            \multicolumn{1}{c|}{\textbf{E3}}                        &
            \multicolumn{1}{c|}{\textbf{Descrição}}                 &
            \multicolumn{1}{c}{\textbf{Avaliação}}

            \\ \hline
            \hline
            \endfirsthead

            \multicolumn{8}{c}%
            {{\bfseries \tablename\ \thetable{} -- continuação da página
                        anterior}}

            \\
            \multicolumn{1}{c|}{\textbf{Identificação do Trabalho}} &
            \multicolumn{1}{c|}{\textbf{I1}}                        &
            \multicolumn{1}{c|}{\textbf{I2}}                        &
            \multicolumn{1}{c|}{\textbf{E1}}                        &
            \multicolumn{1}{c|}{\textbf{E2}}                        &
            \multicolumn{1}{c|}{\textbf{E3}}                        &
            \multicolumn{1}{c|}{\textbf{Descrição}}                 &
            \multicolumn{1}{c}{\textbf{Avaliação}}

            \\ \hline
            \hline
            \endhead

            \hline \multicolumn{8}{r}{{Continua na próxima página}}

            \\
            \endfoot
            \hline \hline
            \endlastfoot

            \bibentry{ahmad_comprehensive_2022}                     &
                                                                    & X &  &  &  & O estudo tem como objetivo investigar e analisar a mutação d 157 genomas de SARS-Cov-2 e suas variantes Delta e Omicron.                                                                                                            & O estudo apresenta a utilização do método Neighbor-Joining que foi utilizado para inferir a história evolutiva. O método não necessita de treinamento.                                                                  \\
            \hline
            \bibentry{yin_systematic_2019}                          &
                                                                    & X &  &  &  & O estudo apresentou a reconstrução a filogenia do HBV com base em 4.429 sequências completas.                                                                                                                                       & O estudo apresenta a utilização da inferência bayesiana sem treinamento.                                                                                                                                                \\
            \hline
            \bibentry{lichtblau_alignment-free_2019}                &
                                                                    & X &  &  &  & O estudo apresentou a utilização de métodos livres de alinhamento de comparação genômica para escalonamento de grandes conjuntos de dados de sequências de nucleotídeos.                                                            & Foi utilizado o método Frequency Chaos Game Representation (FCGR) que cria imagens a partir das sequências de nucleotídeos. O método necessita de treinamento com redes neurais.                                        \\
            \hline
            \bibentry{cho_analysis_2022}                            &
            X                                                       & X &  &  &  & O estudo teve como objetivo confirmar o padrão geral de uso de códons e explorar as características evolutivas e genéticas comumente ou especificamente expressas em HIV1, HIV2 e SIV.                                              & Foi construída árvores filogenéticas e índices do uso de códons. O processo foi realizado cinco vezes para cada gene e foi gerada uma árvore representativa com alto grau de concordância.                              \\
            \hline
            \bibentry{paez-espino_diversity_2019}                   &
                                                                    & X &  &  &  & O trabalho examinou uma coleção com 14.000 metagenomas, identificando 44.221 sequências de virófagos, das quais 328 era genomas completos ou quase completos. Nesses foi realizada a analise genômica comparativa.                  & O alinhamento foi realizado através do MAFFT e as árvores foram construídas com o software Fasttree v2.1.                                                                                                               \\
            \hline
            \bibentry{tang_evolutionary_2021}                       &
                                                                    & X &  &  &  & O estudo analisou variantes de nucleotídeo único (SNVs) em 121.618 genomas de SARS-CoV-2 de alta qualidade.                                                                                                                         & Foi utilizado o método de máxima parcimônia sobre o alinhamento de múltiplas sequências de SARS-CoV-2                                                                                                                   \\
            \hline
            \bibentry{fall_genetic_2021}                            &
                                                                    & X &  &  &  & O estudo analisou a diversidade genética e a dinâmica evolutiva do HSRV no Senegal com dados coletados entre 2008 e 2018 com o objetivo de compreender a base da evolução molecular das cepas.                                      & Foi utilizado o método de máxima verossimilhança. Também foi realizada a análise filodinâmica através do método Markov Chain Monte Carlo com o software Beast.                                                          \\
            \hline
            \bibentry{hudu_hepatitis_2018}                          &
                                                                    & X &  &  &  & O estudo caracterizou a análise genômica comparativa do HEV de 82 pacientes nos anos de 2015 e 2016.                                                                                                                                & Foi utilizado o método de máxima verossimilhança.                                                                                                                                                                       \\
            \hline
            \bibentry{bedoya-pilozo_molecular_2018}                 &
                                                                    & X &  &  &  & O estudo analisou a evolução das variantes do HPV mais prevalentes com base em 166 amostras caracterizando-as através da filogenia e coalescência.                                                                                  & Foi construída árvores filogenéticas com o método Bayesiano através do software BEAST.                                                                                                                                  \\
            \hline
            \bibentry{kim_ngs_2022}                                 &
            X                                                       & X &  &  &  & O estudo propôs métodos para vetorizar os dados da sequência, realizar análises de agrupamento e visualizar os resultados com métodos de aprendizagem de máquina.                                                                   & O método proposto utiliza aprendizagem de máquina com treinamento supervisionado e pode lidar com uma variedade de sequências de dados, podendo ser usado para todos os tipos de doenças, incluindo gripe e SARS-CoV-2. \\
            \hline
            \bibentry{potdar_phylogenetic_2021}                     &
                                                                    & X &  &  &  & O estudo fornece uma integração das classificações filogenéticas existentes, e descreve as tendências evolutivas das cepas de SARS-CoV-2 que circulam na Índia. Foi realizada a análise de 3.014 sequências indianas de SARS-CoV-2. & Foi realizada a análise filogenética das sequências dos genomas através do software MEGA com base na abordagem de junção de vizinhos com a probabilidade composta como modelo de substituição.                          \\
            \hline
            \bibentry{behl_threat_2022}                             &
                                                                    & X &  &  &  & O estudo apresenta uma análise evolutiva da doença infecciosa através de análises filogenéticas.                                                                                                                                    & Foi utilizado o método de máxima verossimilhança.                                                                                                                                                                       \\
            \hline
            \bibentry{sallard_tracing_2021}                         &
                                                                    & X &  &  &  & O estudo apresentou uma discussão bre a origem, natural ou sintética, do SARS-CoV-2, com base em inferências filogenéticas, análises de sequências e relações estrutura-função das proteínas do coronavírus.                        & Foi utilizado inferências filogenéticas, análise de sequências e relações estrutura-função das proteínas.                                                                                                               \\
            \hline
            \bibentry{dimitrov_updated_2019}                        &
                                                                    & X &  &  &  & O estudo propôs um sistema de classificação para facilitar a nomenclatura em estudos da evolução e epidemiologia de vírus da doença de Newcastle.                                                                                   & Foi utilizado os métodos de junção de vizinhos, máxima verossimilhança e Bayesiano.                                                                                                                                     \\
            \hline
        \end{longtable}

    \end{center}

\end{landscape}
\end{document}